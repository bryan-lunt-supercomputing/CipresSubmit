\section{Template Files\label{templatefile}}

The current version of \theprog uses the Python implementation of the \href{http://mustache.github.io/}{{\bf Moustache}} template engine. (Jinja2 , which allows more programming logic in templates, is considered for a future version.)\\
Before actually submitting a job, one of the main responsibilities of \theprog is to evaluate these templates and produce the actual run scripts.

\subsection{Cascading Preferences : Order of Priority} \hfill \\
\theprog sets up parameters for the templates by cascading preferences from the various input files, command-line, and config files, in the following order. (Later entries have stronger precedence)

\begin{itemize}
	\item Template $<$param ... /$>$ entries given in the resource XML file.
	\item Global settings from the .conf files (themselves already cascaded.)
	\item The accountstr entry from the resource XML file. (As 'account')
	\item The URL given at the command-line
	\item All entries from the \filename{\_JOBINFO.TXT} file.
	\item Template $<$env ... /$>$ entries. (Including an additional entry `env\_vars\_string' which joins all variables like: `QOS=2,FOOBAR=test,BAZ=another' )
	\item The Account string from the commandline, if any. (In the future, preferably replaced by putting in \filename{\_JOBINFO.TXT})
	\item The scheduler properties, both read from \filename{scheduler.conf} and those computed in the queue selection step. (`queue', `nodes', `ppn', `runminutes')
	\item The command line to run, as `cmdline'.
\end{itemize}

\subsection{Template Evaluation Environment}
Templates are evaluated by passing them a dictionary or namespace-like object. Fields / keys in this object can be used in the template for substitution and limited looping.
If top-level object A has member B which has member C, then ``B.C" selects A.B.C if A is passed as the parameter to the template.
For example, the name of the selected queue is accessed as ``queue.name".


In addition to all entries documented in \filename{scheduler.conf} and in \filename{\_JOBINFO.TXT}, templates may expect the following:
\begin{description}
	\item[cmdline] The command line to execute.
	\item[account] The account string to bill computation time to.
	\item[queue] An queue object, from the descriptor XML files.
		\begin{description}
			\item[name] The name on the submit-host of the queue to which this job should be submit. (IE \begin{verbatim}#PBS -q <% queue.name %>\end{verbatim} )
			\item[node\_properties] Properties to be used when submitting to this queue.
		\end{description}
	\item[JobHandle] The name of the job according to the portal. Useful as the scheduler job-name
	\item[runminutes] The number of minutes the job should be allowed to run
	\item[nodes] The number of nodes to use.
	\item[ppn] The number of CPUs per node to request.
	\item[env\_vars\_string] String joining all of the template $<$env ... /$>$ entries like: `QOS=2,FOOBAR=test,BAZ=another' (Because Moustache doesn't support much logic.)
	\item[cluster\_header] Could be null, from the resource XML
	\item[jobdir] The current working directory from \theprog, the directory in which the job should be executed.
	\item[job\_status\_email] Where to send e-mail when a job finishes.
	\item[CIPRESNOTIFYURL] URL to connect to to notify the front-end of changes in job status.
\end{description}

Templates need not use all of these. For examples please see \filename{src/CipresSubmit/templates/runfile.template} and \filename{src/CipresSubmit/templates/cmdfile.template}.
