\section{Configuration files}
Various configuration files control the global behaviour and default values of \theprog .

\subsection{.conf files}
Chief among the global configuration files are the python config files ``~/.cipressubmit.conf" and the current directory file ``submit.cfg" . (The current directory file may be removed for security reasons.)

\theprog cascades these file, with ``submit.cfg" overriding values from ``~/.cipressubmit.conf''. An example configuration file is thus:
~\\
\begin{verbatim}
[hosts]
resourcexmldir=/projects/ps-ngbt/submit/descriptors
\end{verbatim}
~\\
The ``resourcexmldir" parameter should point to a directory that contains Submit Resource Descriptor XML files (\ref{resourcexml}). It should give an absolute path requiring no expansion.

\subsection{Resource XML Descriptors\label{resourcexml}}
These files each describe a single resource that \theprog may submit a job to.
The `name' attribute of a resource element should match the `resource' property described in  \ref{jobinfo}

The XML Schema for these files is available in the project under\\ \filename{src/CipresSubmit/schema/resource.xsd}.

\lstset{language=XML}
\begin{lstlisting}[caption={General layout of a resource descriptor}]
<?xml version="1.0" encoding="UTF-8"?>
<resource name="gordon" xmlns="http://www.phylog.org/submit/resource">
	<hosts>
	   It is unnecessary to use this option in the current formulation.
	</hosts>
	<account>
		<accountstr>TG-DEB090011</accountstr>
	</account>
	<batch-system type="PBS">
		<templates>
	Templates go here, the first template is the one that gets qsubbed.
		</templates>
		<queues>
			Queues go here, there should at least be
a queue with the id "normal".  It is suggested that there be a queue with the
id "shared" for jobs that use less than a whole node.
		</queues>
	</batch-system>
</resource>
\end{lstlisting}

\subsubsection{Template Entries}
A template element has a `name' giving the name of the file to be created when this template is executed.
The attribute `filename' gives the (preferably absolute) path to the template file (Sec.~\ref{templatefile}) to be loaded.\\
Each PARAM entry has a `name' giving the name of the parameter as it will appear to the template when that template is evaluated. The internal text is used as a literal string, no substitution is possible.

{\bf Important :} The first template entry is for the script file that will actually be {\it qsubbed}.

\begin{lstlisting}[caption={An example template element}]
<template name="batch_command.run" filename="/Path/To/Templates/runfile.template">
	<param name="cluster_header">
source /etc/profile.d/modules.sh
export MODULEPATH=/home/diag/jpg/modulefiles/gordon/applications:$MODULEPATH
	</param>
</template>
\end{lstlisting}

\subsubsection{Queue Entries}
A queue's ``id" will be the id that the front-end asks for if it specifies a queue. `name' is the queue name on the submission host itself.
In general, it may be best to use ids and names that are identical, with the following exceptions:

\begin{itemize}
	\item It is required to have a queue with id `normal' ; this will run jobs that are node-exclusive or multi-node jobs.
	\item It is best to also have a queue with id `shared' ; this will run jobs that use fewer than one node, IE, if your installation allows running on 2 cpus out of 16. (In some cases, this may point to the same queue as 'normal'.) \\ If missing, all jobs will get run on the `normal' queue, and each job will use at least 1 node.
\end{itemize}

Each {\it env } entry gives the name and value of an environment variable that should be set while submitting the job.
Each {\it property } entry gives a property that should be added to the node-selector at submit time.

\begin{lstlisting}[caption={Example Queue entry}]
<queue id="shared" name="host_name_for_shared_queue">
	<cores-per-node>16</cores-per-node>
	<cores-increment>8</cores-increment>
		<!-- cores-increment defaults to cores-per-node if missing -->
	<max-run-hours>336.0</max-run-hours>
	<env-vars>
		<env name="QOS">2</env>
	</env-vars>
	<node-properties>
		<property>native</property>
		<property>flash</property>
	</node-properties>
</queue>
\end{lstlisting}